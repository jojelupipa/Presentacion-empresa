%%
% Plantilla de Presentación
% Modificación de una plantilla de Latex de LaTeXTemplates para adaptarla 
% al castellano y a las necesidades de escribir informática y matemáticas.
%
% Editada por: Mario Román
%
% License:
% CC BY-NC-SA 3.0 (http://creativecommons.org/licenses/by-nc-sa/3.0/)
%%

%%%%%%%%%%%%%%%%%%%%%
% Beamer Presentation
% LaTeX Template
% Version 1.0 (10/11/12)
%
% This template has been downloaded from:
% http://www.LaTeXTemplates.com
%
% License:
% CC BY-NC-SA 3.0 (http://creativecommons.org/licenses/by-nc-sa/3.0/)
%
%%%%%%%%%%%%%%%%%%%%%

%----------------------------------------------------------------------------------------
%	PAQUETES Y CONFIGURACIÓN DEL DOCUMENTO
%----------------------------------------------------------------------------------------

\documentclass[compress, aspectratio=169]{beamer} % Beamer
\usepackage[spanish]{babel} % Traducciones
\usepackage[utf8]{inputenc} % Uso de caracteres UTF-8
\usepackage[T1]{fontenc} % Permite copiar código y evita errores
\uselanguage{Spanish} % Traducciones beamer
\languagepath{Spanish} % (tex.stackexchange.com/questions/168208)
\usepackage{pgfpages} % Beamer User Guide sections 19.6 and 22
\usepackage[absolute,overlay]{textpos} % Especifica posición del texto.
\usepackage{verbatim} % Bloques de comentarios

%% Temas %%
% Tema y tema de color
\usetheme{Singapore}
\usecolortheme{beaver}

% Fuentes de tamaño arbitrario
\usepackage{lmodern}

% Gráficos
\usepackage{graphicx} % Allows including images
\usepackage{booktabs} % Allows the use of \toprule, \midrule and \bottomrule in tables

%----------------------------------------------------------------------------------------
%	TÍTULO
%----------------------------------------------------------------------------------------

%% Título y otros %%
\title[Johnson \& Johnson]{Exposición sobre Johnson \& Johnson (JNJ) y su unidad estratégica} % The short title appears at the bottom of every slide, the full title is only on the title page

\author[Óscar Bermúdez, Jesús]{
	Óscar Bermúdez Garrido
	(\href{http://www.github.com/oxcar103}{@oxcar103})\\ 
	Jesús Sánchez de Lechina Tejada
	(\href{http://www.github.com/jojelupipa}{@jojelupipa})\\ 
} % Your name

\institute[UGR] % Your institution as it will appear on the bottom of every slide, may be shorthand to save space
{
  Universidad de Granada \\ % Your institution for the title page
}
\date{\today} % Date, can be changed to a custom date

\begin{document}

% Diapositiva de título.
\begin{frame}
	\transdissolve[duration=1]

	\titlepage % Print the title page as the first slide
\end{frame}


%----------------------------------------------------------------------------------------
%	PRESENTACIÓN
%----------------------------------------------------------------------------------------
 
%------------------------------------------------

\section{Introducción}
	\begin{frame}
	\transdissolve[duration=1]

		\frametitle{\insertsection}
		Para comenzar y situarnos en el concepto de estrategia necesitamos saber cuál es el campo de
		actividad de Johnson \& Johnson.
	\end{frame}


\section{Campo de actividad y unidades estratégicas de negocios}

	\begin{frame}
	\transdissolve[duration=1]

		\frametitle{\insertsection}
		
		\only<1->{J\&J es una empresa fabricante de dispositivos médicos, productos farmacéuticos, de
		cuidado personal y para bebés.}
		
		
		%\uncover<3->{A priori puede parecer que todos los productos están muy relacionados entre sí y
			%enfocados a los cuidados o a la higiene, pero esto no hace más que aportar una seguridad al
		%cliente cuando busque una referencia entre la diversidad de productos que puede requerir.}
		
		%\uncover<4>{Además, J\&J se especializa en muchos productos concretos tales como medicamentos y
		%primeros auxilios. Lo cual le hace especializarse en el amplio sector de los “cuidados personales”.}
		%\end{comment}
	\end{frame}
	
	\subsection{Campo de actividad}
	\begin{frame}
	\transdissolve[duration=1]
	
	

	\only<1>{\begin{figure}
			\includegraphics[width=8cm]{Thief.png}  %// Diapositiva 2: Una imagen del logo de JNJ con flechitas de los productos que ofrece 
		\end{figure}}
		
	\end{frame}

	\subsection{Subdivisión de clientes}
		

\section{Criterios de subdivisión de las unidades estratégicas}
\begin{comment}
	\begin{frame}
	\transdissolve[duration=1]

			\frametitle{\insertsubsection}
			La naturaleza de los productos de J\&J hace que amplíe su rango de clientes a prácticamente cualquier
			persona, pues si bien los productos y utensilios médicos sólo les harán falta a médicos y enfermos
			y los productos para bebés sólo a sus cuidadores, pero todo el mundo usa (o puede usar) productos
			de higiene a diario.
			
			\pause
			
			Esto le supone un beneficio doble: Por un lado genera los beneficios correspondientes a ese sector
			y por otro le afianza como referente en el sector para cuando quieran dar el salto a otro sector
			relacionado y aumentar así sus unidades estratégicas.
		\end{frame}
\end{comment}
	\begin{frame}
	\transdissolve[duration=1]

		\huge\centerline{\insertsection}
	\end{frame}
	
	\subsection{Productos paramédicos (BAND-AID bandages)}
		\begin{frame}
	\transdissolve[duration=1]

			\frametitle{\insertsubsection}
			
			\begin{figure}
				\includegraphics[width=12cm]{band.jpeg}
			\end{figure}
			
			%Capta clientes que estén interesados en cubrir los primeros auxilios, de modo que así afectará
			%desde especialistas médicos o socorristas a cualquier persona común que pueda necesitarlo como
			%un excursionista.
			
			%J\&J se mantiene en este negocio gracias al renombre que le aporta su longevidad y a la calidad
			%de sus productos.
			
			
		\end{frame}

	\subsection{Fármacos y otras medicinas (Tylenol)}
		\begin{frame}
	\transdissolve[duration=1]

			\frametitle{\insertsubsection}
			
			\begin{figure}
			   \center\includegraphics[width=8cm]{tylenol.jpeg}			
			\end{figure}
			
			%Para personas alérgicas o con resfriados que involucren pequeños dolores y fiebre este producto
			%actúa como un analgésico y antipirético para rebajar los síntomas. De este modo busca su
			%clientela en médicos, particulares y por supuesto distribuidores y farmacias.
			
			%En esta unidad estratégica cuenta con la diferenciación del producto como una pequeña ventaja
			%aparte de lo ya comentado anteriormente. Además las grandes barreras de entrada en el mercado
			%farmacéutico permiten que estos se afiancen y hayan elegido esta unidad.
		\end{frame}
	
	\subsection{Productos Johnson's baby}
		\begin{frame}
	\transdissolve[duration=1]

			\frametitle{\insertsubsection}
			
			\begin{figure}
			   \center\includegraphics[width=8cm]{baby.jpeg}				
			\end{figure}
			
			%Un conjunto de productos básicos cuya finalidad es mejorar la calidad de los cuidados de los
			%recién nacidos y facilitar el trabajo a sus cuidadores.
			
			%Una vez más, la popularidad de su marca le permite estar en lo más alto. A esto se le suma el
			%hecho de que no hay gran cantidad de competidoras en cuidados de bebés con el mismo potencial
			%para competir.
		\end{frame}
		
		\subsection{Clean \& Clear}
			\begin{frame}
	\transdissolve[duration=1]

				\frametitle{\insertsubsection}
				
			\begin{figure}
			   \center\includegraphics[width=7cm]{clean.jpeg}				
			\end{figure}
			
				%Productos de belleza que siguen la línea de Neutrógena, otra de sus marcas, pero con una
				%especialización del cuidado facial.
				
				%Se introdujo en 1956, y con el auge de la publicidad y el cine consiguieron crecer gracias
				%a la imposición de los cánones de belleza que estos introdujeron. Los cuales se han mantenido
				%y ha permitido que productos como estos sigan siendo buscados por los clientes a día de hoy.
			\end{frame}
	
\section{Referencias}
	% Bibliografía
	\begin{frame}
	\transdissolve[duration=1]

		\frametitle{\insertsection}
		
		Para la realización de estas diapositivas, se han utilizado los repositorios de GitHub:
		
		\footnotesize{
		\begin{thebibliography}{7} % Beamer does not support BibTeX so references must be inserted manually as below
			\bibitem{JnJ} Johnson \& Johnson
				\newblock We've been caring for people for more than 130 years.
				\newblock \url{https://www.jnj.com/}
			\bibitem{Wiki} Wikipedia
				\newblock Wikipedia, the free encyclopedia
				\newblock \url{https://en.wikipedia.org/wiki/Johnson\_\%26\_Johnson}
			\bibitem{Pbaeyens} Pablo Baeyens
				\newblock Guía de uso de beamer
				\newblock \url{https://github.com/dgiim/beamer}
			\bibitem{M42} Mario Román
				\newblock Recopilación de plantillas de Latex.
				\newblock \url{https://github.com/M42/plantillas}
		\end{thebibliography}
		}
	\end{frame}

%------------------------------------------------

\begin{frame}
	\transdissolve[duration=1]

\Huge{\centerline{Fin}}
\end{frame}

%----------------------------------------------------------------------------------------

\end{document}
